\documentclass[border=3mm]{standalone}
\usepackage{tikz}
\usetikzlibrary{calc}
\usetikzlibrary{circuits.logic.US}

\begin{document}
\begin{tikzpicture}[circuit logic US,
                    tiny circuit symbols,
                    every circuit symbol/.style={fill=white,draw,logic gate input sep=4mm, logic gate inverted radius=1mm}
]
\draw (0,0)coordinate (O)--++(90:3)coordinate (A)--++(180:2)coordinate (B)--++(270:3)coordinate (C)--cycle;
\draw ($(C)!0.50!(B)$)--++(0:-1)node[left]{$B$};
\draw ($(C)!0.30!(B)$)--++(0:-1)node[left]{$C$};
\draw ($(A)!0.50!(B)$)++(90:-2mm)node[font=\footnotesize]{2:4};
\draw ($(A)!0.50!(B)$)++(90:-5mm)node[font=\footnotesize]{Decoder};
\foreach \y/\t/\b/\c in {0.1/0/0/0,0.2/1/0/1,0.3/2/1/0,0.4/3/1/1} {
\draw ($(A)! 0.2 + \y*1.5 !(O)$)--++(0:2cm) node[right] (hot\t) {$Y_{\t}$};
\draw ($(A)! 0.2 + \y*1.5 !(O)$)++(0:-0.1) node[left,font=\footnotesize] {$\b$ $\c$};}
\node [or gate, inputs = nn, point down] at (1,-0.75) (or1) {};
\draw (or1.output) -- ++(down:6mm) ++(down:4mm) node[] (Y) {$Y$};
\draw (or1.input 1) |- (hot2);
\draw (or1.input 2) |- (hot3);
\end{tikzpicture}
\end{document}
